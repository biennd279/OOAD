% !TeX root = ../main.tex
\documentclass[./../main.tex]{subfiles}

\begin{document}
% content here
\subsection{Mục đích}
Tài liệu này là báo cáo về chủ đề \textit{Hệ thống hỗ trợ giảng dạy và học tập trực tuyến} của nhóm 06 trong khóa học Phân tích và thiết kế hướng đối tượng.

Báo cáo được viết dựa trên định dạng báo cáo của “IEEE Std 830-1998 IEEE Recommended Practice for Software Requirements Specifications”.

Mục đích của Tài liệu Thiết kế hệ thống con là làm mịn cấu trúc và hành vi của các hệ thống con trong hệ thống.

\subsection{Độc giả dự kiến và đề xuất cách đọc}

Các đối tượng độc giả khác nhau của tài liệu là:
\begin{itemize}
    \item Quản trị dự án: người quản lý và có trách nhiệm về chất lượng của hệ thống. Quản trị dự án nên đọc toàn bộ tài liệu này nhằm phục vụ việc lên kế hoạch và phân công công việc.
    \item Nhà phát triển: người có nhiệm vụ cài đặt hệ thống, chuyển đổi từ bản thiết kế và tài liệu thành phiên bản chạy được. Nhà thiết kế cần đọc tài liệu này để có thể cài đặt hệ thống một cách chính xác.
    \item Người viết tài liệu: những người sẽ viết các tài liệu khác trong tương lai (báo cáo, biên bản họp…). Người viết tài liệu nên đọc và hiểu các biểu đồ ca sử dụng chính.
\end{itemize}

Tài liệu này mô tả thiết kế của hệ thống con. Tài liệu thiết kế hệ thống con dùng để tóm lược hành vi bên trong một “gói” cung cấp những giao diện trực tiếp và gián tiếp và (theo quy ước) không thể hiện bất kỳ nội dung nội bộ nào cho các thành phần khác trong hệ thống. Hệ thống con đươc coi như một đơn vị có hành vi riêng trong hệ thống, được cung cấp các hành vi độc lập và có tương tác với một số lớp/hệ thống con khác.

\subsection{Phạm vi dự án}
\textit{Hệ thống hỗ trợ học tập và giảng dạy trực tuyến Moodle Plus} được xây dựng như một phương tiện hỗ trợ học tập và giảng dạy cho giảng viên và sinh viên.

Hệ thống sẽ được phát triển dưới dạng một ứng dụng web, Người dùng cuối sẽ tương tác với hệ thống qua Internet thông qua các thiết bị thông minh (laptop, PC, máy tính bảng, điện thoại thông minh).

\subsection{Tài liệu tham khảo}
\nocite{*}
\printbibliography

\end{document}