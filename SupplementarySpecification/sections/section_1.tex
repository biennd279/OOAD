\documentclass[./../main_file.tex]{subfiles}

\begin{document}
	\subsection{Giới thiệu}
	
	Tài liệu này xác định những yêu cầu của hệ thống. Danh sách yêu cầu đặc tả bổ sung không bao gồm các ca sử dụng trong mô hình ca sử dụng. Tài liệu đặc tả bổ sung và mô hình ca sử dụng kết hợp với nhau để bao phủ hết các yêu cầu của hệ thống.
	
	Tài liệu được viết dựa theo định dạng báo cáo “IEEE Std 830-1998, IEEE Recommended Practice for Software Requirements Specifications".
	
	\subsection{Đối tượng dự kiến và đề xuất cách đọc}
	
	Các đối tượng đọc khác nhau dành cho tài liệu này là:
	
	\begin{description}
		\item[Người thiết kế] Thiết kế phải đáp ứng được các yêu cầu được chỉ định trong tài liệu này.
		
		\item[Nhà phát triển] Người thực hiện nhiệm vụ phát triển hệ thống từ đầu vào là bản thiết kế và tài liệu để tạo thành đầu ra là một phiên bản có thể chạy được.
		
		\item[Testers] Người có nhiệm vụ đảm bảo rằng các yêu cầu là hợp lệ và phải xác nhận các yêu cầu. Tester nên đọc chi tiết để viết ca kiểm thử phù hợp.
		
		\item[Người dùng] Bất kỳ cá nhân, tư nhân hoặc doanh nghiệp tham gia sử dụng \emph{Hệ thống dạy học trực tuyến Moodle Plus}.
		
		\item[Người viết tài liệu] Người sẽ viết tài liệu trong tương lai (các báo cáo, biên bản)
	\end{description}

	Tài liệu này xác định các yêu cầu phi chức năng của hệ thống, chẳng hạn như độ tin cậy, khả năng sử dụng, hiệu suất và khả năng hỗ trợ cũng như các yêu cầu chức năng phổ biến trong một số ca sử dụng. (Các yêu cầu chức năng được xác định trong Tài liệu đặc tả ca sử dụng).
	
	\begin{description}
		\item[Giới thiệu] Giới thiệu về tài liệu cho người đọc.
		\item[Tổng quan hệ thống] Cung cấp mô tả ngắn gọn, mức cao về Hệ thống khóa học moodle và tham gia sự kiện trực tuyến bao gồm mục tiêu, phạm vi, bối cảnh và khả năng của hệ thống.
		\item[Yêu cầu chức năng] Trong đó chỉ định các yêu cầu chức năng hệ thống theo mô hình ca sử dụng.
		\item[Yêu cầu dữ liệu] Trong đó chỉ định các yêu cầu dữ liệu hệ thống theo các thành phần dữ liệu được yêu cầu.
		\item[Yêu cầu chất lượng] Trong đó chỉ định các yếu tố chất lượng hệ thống cần thiết.
		\item[Các ràng buộc] Tài liệu định nghĩa yêu cầu các ràng buộc về kiến trúc, thiết kế và triển khai trên hệ thống.
		
	\end{description}
	
	\subsection{Phạm vi}
	Tài liệu đặc tả bổ sung này dùng cho hệ thống giảng dạy và học tập trực tuyến. Nó định nghĩa các yêu cầu phi chức năng của hệ thống, ví dụ như độ tin cậy, tính khả dụng, hiệu năng, khả năng hỗ trợ cũng như yêu cầu chức năng chúng áp dụng cho một số trường hợp. (Yêu cầu chức năng là những yêu cầu đặc tả mô hình ca sử dụng)
	
	\subsection{Tài liệu tham khảo}

	% TODO: Biển thêm nhé :v
	
\end{document}