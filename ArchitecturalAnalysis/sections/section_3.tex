\documentclass[./../main_file.tex]{subfiles}

\begin{document}
	\subsection{Biểu đồ}
	\begin{figure}[H]
		\centering
		\resizebox{0.8\textwidth}{!}{\input{./images/Key Abstractions_v1.1.pdf_tex}}
		\caption{Các trừu tượng chính}
	\end{figure}
	\subsection{Định nghĩa}
	\begin{description}
		\item[StudentAccount] Tác nhân sử dụng các dịch vụ dành cho học sinh như tham gia khóa học, nhắn tin, gửi bài tập.
		\item[TeacherAccount] Tác nhân sử dụng các dịch vụ dành cho giáo viên như quản lý khóa học đang dạy, tạo tài liệu lớp học, giao bài tập.
		\item[AdminAccount] Tác nhân sử dụng các dịch vụ dành cho admin như quản lý danh sách khóa học, danh sách người dùng.
		\item[Course] Khóa học được mở ở trên hệ thống. Người dùng học sinh và giáo viên sẽ tương tác với đối tượng này để thực hiện công việc học tập.
		\item[Quiz] Bài kiểm tra được giáo viên đưa ra trong khóa học. Học sinh sẽ thực hiện bài kiểm tra để lấy điểm cho khóa học.
		\item[CourseMaterial] Tài liệu trong khóa học. Tài liệu này được giáo viên tải lên để chia sẻ cho học sinh.
		\item[ForumPost] Bài đăng trong diễn đàn thảo luận của lớp. Thành viên trong lớp (học sinh và giáo viên) đều có thể gửi bài đăng lên diễn đàn.
		\item[ForumComment] Bình luận cho các bài đăng trên diễn đàn. Học sinh và giáo viên đều có thể đăng bình luận.
		\item[Assignment] Bài tập về nhà được giao trong khóa học. Giáo viên sẽ giao bài tập và học sinh sẽ tải lên file để hoàn thành bài.
		\item[Submission] Bài nộp của học sinh cho bài tập trong lớp.
		\item[Message] Tin nhắn được gửi giữa người dùng trên hệ thống.
		\item[BlogPost] Bài đăng trên blog cá nhân của người dùng. Mỗi người dùng đều có một blog cá nhân để họ có thể tùy ý gửi bài đăng.
		\item[Notification] Thông báo về các sự kiện như tin nhắn mới, bài tập mới. Thông báo được gửi bởi hệ thống cho các người dùng liên quan đến sự kiện.
		
	\end{description}
	
\end{document}