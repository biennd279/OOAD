\documentclass[./../main_file.tex]{subfiles}

\begin{document}
\subsection{Nhu cầu đối với hệ thống giảng dạy trực tuyến}
Theo thống kê của UNESCO, tháng 5/2020, dịch Covid-19 ảnh hưởng đến 1,2 tỷ người học ở 156 quốc gia. Chính sách giãn cách xã hội làm bùng nổ giáo dục trực tuyến trên thế giới với các hình thức học khác nhau. Vì vậy, ngày càng nhiều cơ sở giáo dục đang tìm kiếm giải pháp phần mềm cho mục đích học trực tuyến.


Cụ thể, tại Việt Nam, trong giai đoạn cách ly xã hội vì dịch Covid -19, các trường học triển khai nhiều hình thức học khác nhau như: ghi lại bài giảng và chia sẻ trên YouTube; thực hiện giảng trực tuyến qua phần mềm gọi video như Zoom, Skype; gửi thông báo qua Zalo hoặc Facebook. Đây là một nỗ lực đáng ghi nhận của Việt Nam trong giai đoạn chuyển hóa sang dạy học trực tuyến. 


Thế nhưng, giải pháp này vẫn còn nhiều bất cập. Thứ nhất, việc sử dụng quá nhiều ứng dụng và thiếu sự đồng nhất giữa các giáo viên đã khiến nhiều học sinh học gặp khó khăn trong việc ghi nhớ môn nào sử dụng công cụ gì. Thứ hai, nhiều ứng dụng phổ thông chỉ hỗ trợ tiếng Anh nên rất khó sử dụng đối với người không hiểu ngôn ngữ này. Thứ ba, các ứng dụng này đều hướng đến đối tượng người dùng phổ thông, vì vậy chúng không cung cấp các tính năng hướng tới giáo dục.


Hơn nữa, khảo sát trên 205 sinh viên về nhu cầu với các hệ thống công nghệ thông tin phục vụ học trực tuyến, có 80\% sinh viên trả lời cần dịch vụ thư viện online; 82\% sinh viên cần cơ sở dữ liệu về bài giảng; 62\% sinh viên cần hệ thống thông tin quản lý sinh viên; 42\% sinh viên cần góc blog/chat; 30\% sinh viên cần bài kiểm tra điện tử và 52\% sinh viên cho rằng, hệ thống hỗ trợ sinh viên là cần thiết.


Như vậy, nhu cầu của người dùng đối với hệ thống học trực tuyến vẫn chưa được đáp ứng một cách toàn diện. Vì vậy, nhóm đưa ra đề xuất về một hệ thống giảng dạy và học trực tuyến tích hợp để giải quyết các vấn đề nói trên.
\subsection{Khó khăn của các hệ thống hiện thời}
Hệ thống hỗ trợ giảng dạy và học tập trực tuyến trên thị trường có các nhược điểm như sau
\begin{itemize}
	\item Hỗ trợ ít ngôn ngữ: hầu hết các sản phẩm trên thị trường như Remind, Edmodo đều chỉ hỗ trợ ngôn ngữ tiếng Anh. Vì vậy, sản phẩm sẽ khó sử dụng cho đối tượng người dùng trẻ tuổi và người dùng ở các nước không có tiếng Anh.
	\item Ít tính năng: các sản phẩm như Google Classroom chỉ tích hợp một số tính năng liên quan đến quản lý khóa học chứ không có các tính năng học trực tuyến như bài giảng video, ...
	\item Giao diện khó sử dụng: trong quá trình chuyển đổi từ học trên trường sang học tập trực tuyến, một giao diện người dùng thân thiện là quan trọng. Tuy vậy, nhiều hệ thống cũ vẫn có giao diện lộn xộn, không rõ ràng, khiến cho việc điều hướng và sử dụng trở nên khó khăn.
\end{itemize}



\end{document}