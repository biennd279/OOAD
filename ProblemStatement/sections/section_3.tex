\documentclass[./../main_file.tex]{subfiles}

\begin{document}
Như vậy, giải pháp tối ưu nhất cho vấn đề học trực tuyến là một hệ thống được tích hợp đầy đủ chức năng để phục vụ cho mục đích này. Hệ thống sẽ cần có những tính năng quản lý như quản lý khóa học, sắp xếp tài liệu và bài giảng trong lớp, cũng như các tính năng học trực tuyến như làm bài tập, bài thi và học qua video.
\subsection{Mô tả}
\subsubsection{Người dùng học sinh}
Người dùng sẽ cần đăng nhập trước khi sử dụng tính năng hệ thống. Tài khoản người học có thể được cung cấp qua nhà trường hoặc tự đăng ký trên hệ thống. Sau khi đăng nhập, người học sẽ được đưa tới trang chủ, nơi liệt kê toàn bộ khóa học mà họ đang tham gia, lịch học và hạn nộp bài tập cho các khóa học đó. Người dùng có thể tự do điều chỉnh màn hình này để hiển thị nhiều thông tin hơn.


Khi người dùng truy cập vào khó	a học, hệ thống sẽ liệt kê toàn bộ tài liệu và bài giảng đã được chia sẻ bởi giảng viên. Khi có buổi học trực tuyến, hệ thống sẽ báo cho người học qua email hoặc thông báo điện thoại.
\subsubsection{Người dùng giảng viên}
Sau khi đăng nhập, giảng viên sẽ xem được danh sách khóa học mà mình đang quản lý. Khi truy cập vào lớp, giảng viên sẽ có lựa chọn upload file, chia sẻ bài giảng, hiển thị và ẩn tài liệu, thực hiện bài học online.
\subsubsection{Quản trị viên}
Sau khi đăng nhập, quản trị viên có thể xem danh sách tài khoản và danh sách khóa học. Quản trị viên có thể thực hiện thêm, sửa và xóa đối với hai danh sách này.
\subsection{Nhiệm vụ cơ bản}
Hệ thống có những nhiệm vụ cơ bản sau
	\begin{itemize}
		\item Đăng nhập, đăng ký
		\item Sửa đổi thông tin cá nhân
		\item Cho phép học sinh tham gia vào khóa học
		\item Cho phép học sinh xem và tải tài liệu khóa học 
		\item Cho phép học sinh tham gia buổi học trực tuyến
		\item Cho phép giáo viên chia sẻ tài liệu
		\item Cho phép giáo viên tạo buổi học trực tuyến
		\item Cho phép quản trị viên quản lý tài khoản và khóa học
	\end{itemize}
\subsection{Người dùng cuối}
Người dùng cuối của sản phẩm này sẽ là học sinh, giáo viên và quản trị viên. Mỗi vai trò sẽ được cung cấp giao diện tương ứng để họ có thể hoàn thành công việc của mình.
\subsection{Kết luận}
Hiện nay, nhu cầu học trực tuyến đang tăng cao hơn bao giờ hết, nhưng lại chưa có nhiều hệ thống được phát triển để thỏa mãn nhu cầu này. Hệ thống hỗ trợ giảng dạy và học tập trực tuyến được phát triển nhằm thay thế phương pháp học truyền thống, giúp học sinh và giáo viên tiếp tục công việc học tập qua Internet.

	
\end{document}