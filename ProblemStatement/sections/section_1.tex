\documentclass[./../main_file.tex]{subfiles}

\begin{document}
	\subsection{Giới thiệu}
	Đây là tài liệu mô tả vấn đề của \textit{hệ thống hỗ trợ giảng dạy và học tập trực tuyến}.
	Tài liệu này được sử dụng để xác định các đặc tính của miền vấn đề, giải thích và định nghĩa các vấn đề phát sinh từ đó đề xuất giải pháp xử lý.
	
	\subsection{Đối tượng dự kiến và đề xuất cách đọc}
	Tài liệu này hướng đến những đối tượng sau
	\begin{description}
		\item [Nhà phát triển] Người thực hiện việc thiết kế và cài đặt hệ thống từ đầu vào là yêu cầu của khách hàng.
		\item [Khách hàng] Người đặt hàng hệ thống và mong muốn một hệ thống mới tốt hơn hệ thống cũ. Trong khóa học này, có thể coi khách hàng là giảng viên.
		\item [Người viết tài liệu tương lai] Người thực hiện việc cập nhật, bổ sung cho tài liệu dự án trong tương lai.
		
	\end{description}
	Nội dung báo cáo gồm ba phần:
	\begin{description}
		\item [Đặt vấn đề] Giải thích lý do tại sao nhóm chọn hệ thống hỗ trợ giảng dạy và học tập trực tuyến. Nội dung bao gồm: mô tả thực trạng hiện thời, diễn tả hệ thống hiện thời (system-as-is), các vấn đề và khó khăn còn tồn đọng.
		\item [Giải pháp] Theo yêu cầu của người dùng và phân tích người dùng cuối từ phần trước, nhóm đã đề xuất ra một giải pháp. Phần này giúp trả lời câu hỏi: Hệ thống mới (system-to-be) sẽ vận hành như thế nào?
		\item [Người dùng cuối] Mô tả các nhóm người dùng mà hệ thống đang hướng đến. Phần nào giúp trả lời câu hỏi: Hệ thống được xây dựng để phục vụ những đối tượng nào?
	\end{description}

	\subsection{Phạm vi dự án}
	Hệ thống sẽ được phát triển dưới dạng ứng dụng Web và ứng dụng Mobile. Người dùng cuối sẽ cần mạng Internet và một thiết bị thông minh để có thể truy cập và sử dụng hệ thống.

	
	\subsection{Tài liệu tham khảo}
	% TODO: Thêm tài liệu tham khảo
	\nocite{*}
	\printbibliography[heading=none]
	
	\clearpage
	
\end{document}