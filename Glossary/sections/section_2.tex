\documentclass[./../main_file.tex]{subfiles}

\begin{document}
\subsection{Tài khoản (account)}

Một bản ghi về người dùng/quản trị viên chứa thông tin về họ tên, địa chỉ e-mail, mật khẩu, số điện thoại và các thông tin cá nhân tùy chọn. Mỗi tài khoản có một ID người dùng và mật khẩu duy nhất, được sử dụng để xác định người dùng/quản trị viên và cấp cho họ quyền truy cập vào các phần nhất định trong hệ thống.

\subsection{Người dùng (user)}
Bất kỳ người nào có tài khoản đã đăng ký trên hệ thống nhưng không phải là quản trị viên. Người dùng có thể thực hiện nhiều tác vụ trên hệ với tài khoản của họ.

\subsection{Quản trị viên (administrator)}
Người có trách nhiệm đảm bảo rằng hệ thống không có quảng cáo spam hoặc các hành vi lừa đảo, lạm dụng. Điều này đòi hỏi phải phê duyệt nội dung, quảng cáo trước khi chúng được xuất bản, xóa các nội dung được báo cáo và chặn/xóa người dùng có các hành vi lừa đảo, lạm dụng.

\subsection{Khách truy cập (visitor)}
Người không có tài khoản nhưng có thể xem các bài đăng, sự kiện trên hệ thống, bị hạn chế nhiều tính năng và tác vụ trên hệ thống. 

\subsection{Sinh viên (student)}
Là người dùng hệ thống, sử dụng hệ thống để học tập bằng cách tham gia các khóa học, truy cập tài nguyên của khóa học, trả lời các câu hỏi, làm bài tập, tham gia thảo luận trong các diễn đàn, sự kiện,... của khóa học.

\subsection{Giảng viên (teacher)}
Là người dùng hệ thống, sử dụng hệ thống để giảng dạy thông qua việc xây dựng các bài giảng theo chủ đề, đăng tải tài liệu học tập, giao các câu hỏi và bài tập cho sinh viên, tạo ra các diễn đàn trao đổi, sự kiện,... theo nội dung của khóa học.

\subsection{Người tham gia (participants)}
Danh sách tất cả những người dùng có trong một khóa học.


\subsection{Người dùng hoạt động (online users)}
Danh sách những người dùng hiện đang truy cập vào hệ thống.

\subsection{Thông tin cá nhân (profile)}
Bao gồm họ tên đầy đủ, thông tin và địa chỉ liên lạc, các khóa học tham gia, các diễn đàn thảo luận, hoạt động đăng nhập lần đầu tiên và lần gần đây nhất của người dùng...

\subsection{Khóa học (course)}
Là nơi mà giảng viên tạo ra tài liệu và các hoạt động học tập như diễn đàn thảo luận, khóa học chủ đề, sự kiện,... cho sinh viên.

\subsection{Thông báo (notification)}
Danh sách thông tin về các hoạt động cá nhân ví dụ như: lời mời kết bạn, tin nhắn mới,... hoặc các hoạt động có trong khóa học: thông báo đến từ giảng viên, bài tập mới, sự kiện mới, diễn đàn thảo luận,...

\subsection{Huy hiệu (badge)}
Là một danh hiệu được cấp cho sinh viên sau khi sinh viên đã hoàn thành các tiêu chí được đặt ra cho từng huy hiệu.

\subsection{Diễn đàn (forum)}
Là nơi mà người tham gia khóa học có thể cùng nhau thảo luận, trao đổi ý kiến, chia sẻ kiến thức về một chủ đề nào đó cùng quan tâm.

\subsection{Chủ đề (topic)}
Là một vấn đề được giảng viên đưa ra theo từng khoảng thời gian ví dụ như một tuần, để sinh viên có thể theo dõi các bài giảng, tham gia thảo luận, đưa ý kiến cho những vấn đề ấy. 

\subsection{Khóa học chủ đề (workshop)}
Một hoạt động nhỏ trong khóa học được giảng viên tạo ra, thường chỉ tập trung vào một chủ đề nhất định như phương pháp, kỹ năng của một lĩnh vực nào đó.

\subsection{Sự kiện (event)}
Một hoạt động được giáo viên đề xuất vào một khoảng thời gian và địa điểm xác định, nhằm phục vụ một mục đích nhất định và được tổ chức tại một khoảng thời gian và địa điểm xác định.

\subsection{Bài giảng (lesson)}
Bài giảng được nhắc đến ở đây là bài giảng điện tử, là một phần nội dung trong chương trình của một khóa học, được tổ chức dựa vào các thiết bị công nghệ như máy tính, điện thoại, dạy và học thông qua môi trường internet. Mỗi bài giảng có thể có âm thanh lời giảng, hình ảnh, video, hoặc tài liệu văn bản được sắp xếp theo logic giúp người học thu được những kỹ năng, kiến thức nhất định.

\subsection{Câu hỏi (quiz)}
Câu hỏi thường ngắn gọn, được giảng viên đặt ra dựa trên mục đích kiểm tra mức độ hiểu bài của sinh viên, có thể trước bài giảng, ngay trong bài giảng hoặc sau đó. 

\subsection{Bài tập (assignment)}
Bài tập được giảng viên giao cho sinh viên làm để vận dụng những điều đã học, thời gian nộp bài được xác định trước, và có thể là bài tập cá nhân hoặc bài tập nhóm.

\subsection{Bài nộp (submission)}
Bài nộp được sinh viên làm tương ứng với bài tập mà giảng viên giao cho và phải nộp đúng hạn.

\subsection{Ngân hàng nội dung (content bank)}
Ngân hàng nội dung là một kho lưu trữ tài liệu học tập có trong khóa học.

\subsection{Phản hồi khóa học (course feedback)}
Sinh viên trả lời các câu hỏi được tạo ra từ trước nhằm khảo sát mức độ hài lòng sau khi tham gia khóa học.

\end{document}