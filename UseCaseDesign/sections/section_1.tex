% !TeX root = ../main.tex
\documentclass[./../main.tex]{subfiles}

\begin{document}
% content here
\subsection{Mục đích}
Tài liệu này là báo cáo về chủ đề Hệ thống hỗ trợ giảng dạy và học tập trực tuyến moodle plus của nhóm 06 trong khóa học Phân tích và thiết kế hướng đối tượng.

Báo cáo được viết dựa trên định dạng báo cáo của  “IEEE Std 830-1998 IEEE Recommended Practice for Software Requirements Specifications”.

Mục đích của Tài liệu Thiết kế ca sử dụng là làm mịn việc hiện thực hóa ca sử dụng về mặt tương tác, yêu cầu đối với hoạt động của các lớp thiết kế, hệ thống con và/ hoặc giao diện hệ thống của chúng.
Mỗi hiện thực hóa ca sử dụng phải được làm mịn để mô tả các tương tác giữa các đối tượng thiết kế tham gia như sau: xác định từng đối tượng tham gia vào luồng sự kiện sử dụng, biểu diễn từng đối tượng tham gia trong biểu đồ  tương tác (Các hệ thống con có thể được biểu diễn bằng các thể hiện của giao diện của hệ thống con), minh họa thông điệp gửi giữa các đối tượng bằng cách tạo tin nhắn giữa các đối tượng, mô tả những gì đối tượng làm khi nhận được tin nhắn.

Đối với mỗi hiện thực hóa ca sử dụng, nhóm đã minh họa các mối quan hệ giữa các lớp hỗ trợ các cộng tác được mô hình hóa trong các sơ đồ tương tác bằng cách tạo một hoặc nhiều sơ đồ lớp.

\subsection{Độc giả dự kiến và đề xuất cách đọc}

Vai trò Designer: Vai trò của người thiết kế xác định trách nhiệm, hoạt động, thuộc tính và mối quan hệ của một hoặc một số lớp và xác định cách chúng sẽ được tùy chỉnh theo môi trường cài đặt. Ngoài ra, vai trò thiết kế có thể chịu trách nhiệm thiết kế gói hoặc thiết kế hệ thống, bao gồm mọi lớp thuộc sở hữu của gói hoặc hệ thống con.

Các đối tượng độc giả khác nhau của tài liệu là:
\begin{itemize}
    \item Quản trị dự án: người quản lý và có trách nhiệm về chất lượng của hệ thống. Quản trị dự án nên đọc toàn bộ tài liệu này nhằm phục vụ việc lên kế hoạch và phân công công việc.
    \item Nhà phát triển: người có nhiệm vụ cài đặt hệ thống, chuyển đổi từ bản thiết kế và tài liệu thành phiên bản chạy được. Nhà thiết kế cần đọc tài liệu này để có thể cài đặt hệ thống một cách chính xác.
    \item Kiểm thử: người kiểm thử nên đọc tài liệu nhằm mục đích viết các ca kiểm thử đơn vị.
    \item Người viết tài liệu: những người sẽ viết các tài liệu khác trong tương lai (báo cáo, biên bản họp…). Người viết tài liệu nên đọc và hiểu các biểu đồ ca sử dụng chính.
\end{itemize}

\subsection{Phạm vi dự án}
\textit{Hệ thống hỗ trợ học tập và giảng dạy trực tuyến Moodle Plus} được xây dựng như một phương tiện hỗ trợ học tập và giảng dạy cho giảng viên và sinh viên.

Hệ thống sẽ được phát triển dưới dạng một ứng dụng web, Người dùng cuối sẽ tương tác với hệ thống qua Internet thông qua các thiết bị thông minh (laptop, PC, máy tính bảng, điện thoại thông minh).

\subsection{Tài liệu tham khảo}
\nocite{*}
\printbibliography

\end{document}