\documentclass[./../main_file.tex]{subfiles}

\begin{document}
	\subsection{Giới thiệu}
	Tài liệu để xác định các yêu cầu về chức năng của hệ thống đã được phân tích, được xem xét bởi người quản lý và tiếp thị dự án.
	Nội dung tài liệu bao gồm hai phần chính:
	\begin{description}
		\item[Sơ đồ ca sử dụng] Phần đầu tiên là sơ đồ chính của Ca sử dụng của Hệ thống hỗ trợ giảng dạy và học tập trực tuyến, bao gồm: Các tác nhân, ca sử dụng và các mối quan hệ giữa các tác nhân và ca sử dụng.
		\item [Đặc tả Ca sử dụng] Mô tả chi tiết về các ca sử dụng tiêu biểu của hệ thống. Mỗi ca sử dụng có một mô tả tương ứng. Trong phần mô tả cho từng ca sử dụng, nhóm sẽ cung cấp mô tả ngắn gọn, luồng sự kiện (luồng chính và luồng thay thế), các mối quan hệ, các yêu cầu đặc biệt, điều kiện trước, điều kiện sau và điểm mở rộng. Những điều này là đủ để xác định ca sử dụng.
	\end{description}
	\subsection{Đối tượng dự kiến}
	Tài liệu này có thể dành cho các đối tượng
	\begin{description}
		\item [Quản lý dự án] Người phụ trách quản lý và chịu trách nhiệm về chất lượng hệ thống. Quản lý dự án nên đọc toàn bộ tài liệu để phục vụ việc lên kế hoạch và phân công công việc.
		\item [Nhà phát triển] Người thực hiện nhiệm vụ phát triển hệ thống từ đầu vào là bản thiết kế và tài liệu để tạo thành đầu ra là một phiên bản có thể chạy được.
		\item [Testers] Người có nhiệm vụ đảm bảo rằng các yêu cầu là hợp lệ và phải xác nhận các yêu cầu. Tester nên đọc chi tiết để viết ca kiểm thử phù hợp.
		\item [Người viết tài liệu] Người sẽ viết các tài liệu liên quan về hệ thống
	\end{description}
	\subsection{Tài liệu tham khảo}
	\nocite{*}
	\printbibliography[heading=none]
	
\end{document}